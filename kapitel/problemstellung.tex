\section{Problemstellung}
In der Vergangenheit wurde der Wert einer Immobilie durch eine 
Immobilienfirma sowie die Bereitschaft eine bestimmte Miete für 
eine Immobilie durch eine Privatperson zu zahlen an den direkten 
Gegebenheiten der Immobilie selbst und deren unmittelbarer Umgebung 
gemessen. Die Lage einer Immobilie war das höchste Gut bei dem 
Bemessen einer Miete. Jedoch hat sich dieses Verhalten in den 
letzten Jahren verändert, durch das hohe Angebot von etwaigen 
Bewertungsportalen und Kartenwerkzeugen im Internet, wie 
beispielsweise Google, Yelp oder Tripadvisor, welche eine neue 
Sicht auf die makroskopische Betrachtung einer Immobilie 
ermöglichen. Mit diesen Werkzeugen lässt sich heutzutage die 
Lebensqualität messen, welche im Zusammenspiel mit der Lage dem 
potenziellen Mieter etliche neue Variablen für seine Betrachtung 
mitgeben. So kann dieser den Grad der Bereitschaft zum Zahlen einer 
gewissen Miete konkreter bilden und durch den erweiterten Blick auf 
die frei zugängliche makroskopische Ebene neu entwickeln. Zu dem 
kann sich der Wert der Immobilie je nach Zielgruppe sowie 
Quartiersentwicklung unter diesen Variablen sehr differenziert 
weiterentwickeln. Diesen Einfluss der neuartigen Variablen, 
welche auch als nicht-traditionellen Kennwerte bezeichnet werden, 
wollen wir in dieser Arbeit untersuchen.
