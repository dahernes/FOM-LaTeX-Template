%\newpage
\section{Methodik}
Der Aufbau dieser Thesis erfolgt nach Vorbild des Vertiefungsmodells 
nach Phillip Mayring, welcher Bestandteile der quantitativen und 
qualitativen Analysen zu einem kombinierten Gesamtmodell vereint\footcite[Vgl. ][]{Mayring2001}.  
Auf Basis des Vertiefungsmodells wird die quantitative Untersuchung 
in Anlehnung an dem CRISP-DM durchgeführt, einem Standartprozessmodell 
zur Modellierung von Data Mining Projekten im Bereich der 
computerunterstützen Analyse\footcite[Vgl. ][]{Wirth2000}. Die aus dieser Untersuchung 
ermittelten Ergebnisse werden daraufhin vertieft, verglichen sowie diskutiert und 
bilden die qualitative Untersuchung ab.

Im ersten Teil dieser Thesis wird der aktuelle Stand der kommunalen Stadtentwicklung 
sowie der Entwicklung des Wohnungsbedarfs der Stadt Essen dargestellt\footcite[Vgl. ][]{StadtEssen2018}. 
Zudem werden die traditionellen und neuartigen Faktoren definiert und deren Unterschiede 
beschrieben, welche für die Beantwortung der zentralen Fragestellung mitunter 
untersucht werden\footcite[Vgl. ][]{Asaftei2018}.

Die Beschreibung, Aufbereitung und Zusammenführung der Datensätze, bestehend aus den 
verschiedenen Faktoren der traditionellen und neuartigen immobilienwirtschaftlichen 
Faktoren erfolgt im zweiten Teil. Der Hauptschwerpunkt dieses Arbeitsschrittes, liegt 
neben der Beschreibung aller genutzten Datensätze, in der explorativen Datenanalyse sowie 
deren Aufarbeitung für den Einsatz in den Prognosemodellen. Die Aufbereitung umfasst 
verschiedene Methoden von der Datenbereinigung\footcite[Vgl. ][]{Rahm2000}, bis hin zur 
Transformationsmethoden\footcite[Vgl. ][]{Box1964} sowie Encoding Varianten\footcite[Vgl. ][]{Cerda2018}, 
um diese bestmöglich für die Analyse vorzubereiten. Im letzten Abschnitt des zweiten 
Teils wird der gesamte Datensatz für das Training in den Modellen 
vorbereitet. Dieser wird für diesen Zweck in drei Teile aufgeteilt. Einem Trainings- 
sowie Testdatensatz, welche für den Einsatz an den Modellen genutzt werden wird und 
einem zufällig auserwählten Validierungsdatensatz.

Im letzten Teil der Analyse werden nunmehr Prognosemodelle auf Basis der zuvor 
aufgearbeiteten Daten eingesetzt, indem diese mithilfe des Training- sowie Testdatensatzes 
trainiert werden, so dass diese an dem Validierungsdatensatz durchgeführt werden können. 
Die Prognosen des Validierungsdatensatzes bilden einen essenziellen Teil der Diskussion ab. 
Die Optimierung der Hyperparameter, der genutzten Modelle erfolgt mit Hilfe der Random 
Search Methode\footcite[Vgl. ][]{Bergstra2012}.

Folgende Prognosemodelle kommen für die Analyse zum Einsatz:

\begin{itemize}
    \item Lineare Regression\footcite[Vgl. ][]{Verbeek2017}
    \item Random Forest\footcite[Vgl. ][]{Breiman2001}
    \item XGBoost\footcite[Vgl. ][]{Chen2016}
    \item Multilayer Perceptron mit mehr als zwei Ebenen\footcite[Vgl. ][]{Brause1991, Goodfellow2017} 
\end{itemize}

Die Effizienz dieser Modelle werden anhand ihres Gütemaßes dem adjusted R² (kurz adj. R²) 
sowie der Verlustmessung Root Mean Squared Error (kurz RSME) bewertet\footcite[Vgl. ][]{Goodfellow2017,Larose2015}.
Zudem können auf Basis der erstellten Prognosen die einflussreichsten Faktoren aller 
Modelle entnommen werden. Die Effizienz sowie die Faktoren der Modelle, welche einen 
hohen Einfluss die Prognose haben sind weitere Bestandteile der Ergebnisdiskussion neben 
den Prognosen.
\newpage
Validiert werden die Ergebnisse anhand zweier Benchmarks. Diese sind zu einem der 
ortsübliche Mietspiegel\footcite[Vgl. ][]{AmtfurStadterneuerungundBodenmanagement-StadtEssen2020}, 
der durch die Stadt Essen erhoben worden ist, sowie den 
ermittelten Angebotsmieten der Zeiträume zweites Quartal 2018 bis erstes Quartal 2019, 
welcher durch die CBRE, einen internationalen Dienstleister aus der Immobilienwirtschaft, 
im LEG-Wohnungsmarktreport 2019 erhoben worden \footcite[Vgl. ][]{LEG2019}.  

Nach der Analyse werden die Prognosen, die Effizienz der Modelle sowie die einflussreichsten 
Faktoren im fünften Teil diese Thesis in der Ergebnisdiskussion vertieft, indem diese 
übersichtlich aufgearbeitet und dementsprechend gegenübergestellt werden. 
