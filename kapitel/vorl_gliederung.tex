\newpage
\section{Vorläufige Gliederung}

\begin{table}[H]
    \begin{tabularx}{\textwidth}[ht]{|l|X|X|}
      \hline
      \textbf{Nr.} & \textbf{Kapitel} & \textbf{Behandelt}\\
      \hline
      \hline
        1 & \textbf{Einleitung}

        \begin{itemize}
            \item 1.1 Motivation
            \item 1.2 Problemstellung
            \item 1.3 Zielsetzung
        \end{itemize} 
        
        & \begin{itemize}
            \item Ziel- und Problembeschreibung des Themas
        \end{itemize}\\
        \hline\hline
        2 & \textbf{Der Immobilienmarkt in Essen, NRW} 
        
        \begin{itemize}
            \item 2.1 Aktuelle und prognostizierte kommunale Entwicklungen der Mietsituation 
            und des Mietbedarfes der Stadt Essen
            \item 2.2 Beschreibung der Unterschiede der traditionellen und 
            nicht-traditionellen Faktoren, welche die Grundlage der Analyse bilden
        \end{itemize}

        & \begin{itemize}
            \item Grundlagenbeschreibung
            \item Spiegelt den Bereich „Business Understanding“ der CRISP-DM Methology ab
        \end{itemize}\\
        \hline\hline
        3 & \textbf{Der Datensatz und dessen Aufbereitung} 
        
        \begin{itemize}
            \item 3.1 Beschreibung der Datensätze sowie der Benchmarks, welche zur 
            Analyse genutzt werden
            \item 3.2 Beschreibung der Datenbereinigung und Zusammenschluss der 
            traditionellen und nicht-traditionellen Faktoren
            \item 3.3 Explorative Datenanalyse
            \item 3.4 Präparation der Daten für die Anwendungen zur Prognose
        \end{itemize}
        
        & \begin{itemize}
            \item Beschreibung und Aufarbeitung aller genutzten Daten, welche die 
            Basis für die Prognosen bilden
            \item -	Spiegelt die Bereiche „Data Understanding“ sowie 
            „Data Preparation“ der CRISP-DM Methology ab
        \end{itemize}\\
        \hline
        \multicolumn{3}{r}{\textit{Fortführung auf der nächsten Seite}} \\
    \end{tabularx}\\
    \end{table}