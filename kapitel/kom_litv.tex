\section{Kommentiertes Literaturverzeichnis}
\fullcite{AmtfurStadterneuerungundBodenmanagement-StadtEssen2020}
\begin{itemize}
    \item Der Mietspiegel 2020 der Stadt Essen ist ein Leitfaden zur 
    Ermittlung von Mietrichtwerten der Stadt Essen und deren 
    Wohnimmobilien.
\end{itemize}

\fullcite{Asaftei2018}
\begin{itemize}
    \item Diese Untersuchung, welche durch McKinsey \& Company ausgeführt
    worden ist, behandelt die nicht-traditionellen immobilienwirtschaftliche
    Kennwerte und betrachtet diese auf dem amerikanischen Immobilienmarkt. 
\end{itemize}

\fullcite{Bergstra2012}
\begin{itemize}
    \item Der Random Search ist ein Vorgehen, welches die optimalsten
    Parameter eines Modells ermittelt mit dem Ziel die besten Schätzer
    zu erhalten. Durch diesen Algorithmus werden vorher festgelegte
    Reichenweiten von Parametern durchgetestet und anhand einer
    Verlustmessung bewertet und auserwählt.
\end{itemize}
\newpage
\fullcite{Box1964}
\begin{itemize}
    \item Die BoxCox-Transformation ist eine beliebte Methode für die
    Transformation von Daten in eine Normalverteilung. In dieser
    wissenschaftlichen Arbeit wird die Umsetzung dieser Methodik 
    beschrieben und mit anderen Tranformationmethoden verglichen.
\end{itemize}

\fullcite{Brause1991}
\begin{itemize}
    \item Rüdiger Brause beschreibt in diesem Buch die 
    naturwissenschaftlichen Grundlagen von neuronalen Netzen und 
    die daraus entstandenen Modellarchitekturen in der Informatik.
\end{itemize}

\fullcite{Breiman2001}
\begin{itemize}
    \item Der Random Forest, welcher ein gängiger Algorithmus aus 
    dem Bereich des Machine Learning ist wird in dieser 
    wissenschaftlicher Arbeit in Gänze beschrieben. Dieser Algorithmus 
    schätzt die Prognose mit Hilfe von Entscheidungsbäumen.
\end{itemize}

\fullcite{Cerda2018}
\begin{itemize}
    \item Wenn in einem Datensatz kategoriale Daten vorhanden sind, ist es für 
    diverse Analysemethoden notwendig, dass diese in angepasste Merkmalsvektoren 
    (eng. feature vector) in der Regel in nominale Werte umgeschrieben werden.
    Dies wird über Encodingmethoden ermöglicht, welche in dieser Untersuchung
    beschrieben werden.
\end{itemize}

\fullcite{Chen2016}
\begin{itemize}
    \item Der XGBoost Algorithmus ist eine moderne Erweiterung der 
    Entscheidungsbäume/Ensemble für Prognosen im Bereich des Machine Learning und 
    erfreut sich hoher Beliebtheit. Tianqi Chen und Carlos Guestrin sind die Erfinder 
    dieses Algorithmus und beschreiben diesen in dieser Arbeit in voller Gänze.
\end{itemize}
\newpage
\fullcite{Goodfellow2017}
\begin{itemize}
    \item Dieses Buch behandelt die Thematik rund um Deep Learning. In diesem werden die
    mathematischen und konzeptionellen Hintergrunde, relevante Konzepte der 
    linearen Algebra und Wahrscheinlichkeitstheorie beschrieben, welche gebraucht werden um neuronalen
    Netze aufzubauen. Neben diesen Grundlagen geht dieses Buch noch weiter und beschreibt zu dem
    die Optimierung und Regulierung der diversen Arten von neuronale Netzen.
\end{itemize}

\fullcite{Larose2015}
\begin{itemize}
    \item Data Mining findet täglich mehr Verbreitung, denn es ermöglicht Unternehmen 
    profitable Muster und Trends aus ihren bestehenden Datenbanken aufzudecken und diese zu
    nutzen. Daniel T. und Chantal D. Larose beschreiben verschiedene Analysemethoden für
    ein effizientes Data Mining.
\end{itemize}

\fullcite{LEG2019}
\begin{itemize}
    \item Die LEG Immobilien AG bringt in regelmäßigen Abständen in Zusammenarbeit mit der 
    CBRE den LEG-Wohnungsmarktreport heraus. Dieser Report analysiert die aktuelle Situation 
    der Wohnimmobilienmärkte bezogen auf die Großstädte der Bundesrepublik Deutschland.
\end{itemize}

\fullcite{Mayring2001}
\begin{itemize}
    \item In dieser Arbeit stellt Phillip Mayring verschiedene Hybridmodelle für empirische
    Untersuchungen vor. Das Vertiefungsmodell, welcher der Rahmen dieser Thesis ist wird
    darin beschrieben.
\end{itemize}

\fullcite{Rahm2000}
\begin{itemize}
    \item Die Datenbereinigung gehört zu den zeitintensivsten 
    Schritten in der Modellentwicklung. In dieser Arbeit werden Methoden 
    und Ansätze von Lösungsanätzen zu wiederkehrenden Problemen in der 
    Datenbereinigung beschrieben. 
\end{itemize}
\newpage
\fullcite{StadtEssen2018}
\begin{itemize}
    \item Dieses Gutachten, welches durch die Stadt Essen in Auftrag gegeben worden ist, 
    behandelt die Frage nach dem Wohnungsangebot sowie -nachfrage in den kommenden Jahren 
    und skizziert mögliche Handlungsfelder um den steigenden Bedarf zu minimieren. 
\end{itemize}

\fullcite{Verbeek2017}
\begin{itemize}
    \item Dieses Buch dient als Leitfaden für alternative Techniken in der Ökonometrie mit 
    Schwerpunkt auf Intuition und der praktischen Umsetzung dieser Ansätze. Es deckt 
    ein breiten Themenspektrum ab einschliesslich der linearen Regression mit dessen 
    Diagnostik, Zeitreihenanalysen und Paneldatenanalysen. 
\end{itemize}

\fullcite{Wirth2000}
\begin{itemize}
    \item Diese wissenschaftliche Arbeit beschreibt den kompletten CRISP-DM Prozess 
    mit allen zugehörigen Prozessschritten und deren Inhalten. 
\end{itemize}