\newpage
\begin{table}[H]
    \begin{tabularx}{\textwidth}[ht]{|l|X|X|}
      \hline
      \textbf{Nr.} & \textbf{Kapitel} & \textbf{Behandelt}\\
      \hline
      \hline
        4 & \textbf{Modellierung und Analyse} 
        
        \begin{itemize}
            \item 4.1 Beschreibung der verwendeten Modelle sowie deren Einsatz zur 
            Prognose der Mieten 
            \begin{itemize}
                \item Lineare Regression
                \item Random Forest
                \item XGBoost
                \item Multilayer Perceptron
            \end{itemize}
            \item 4.2 Vergleich der Güte sowie der Verlustmessungen
        \end{itemize}
        
        & \begin{itemize}
            \item Weiterverarbeitung aller genutzten Daten, sowie eine erste 
            Analyse, durch den Einsatz der Prognosemodelle
            \item Spiegelt den Bereich „Modeling“ der CRISP-DM Methology ab
        \end{itemize}\\
        \hline\hline
        5 & \textbf{Evaluierung und Diskussion} 
        
        \begin{itemize}
            \item 5.1 Aufbereitung der Validierungsergebnisse mit den verschiedenen 
            Prognosen sowie Gegenüberstellung mit dem Benchmark
            \item 5.2 Betrachtung der effizientesten Faktoren, welche durch die 
            Prognosemodelle für die Validierung genutzt worden sind
            \item 5.3 Diskussion über die Ergebnisse
        \end{itemize}

        & \begin{itemize}
            \item Behandelt die Validierung und diskutiert diese mit Betrachtung auf 
            die genutzten Prognosemodelle und vergleicht diese mit den Benchmarks, sowie 
            deren Faktoren
            \item Spiegelt den Bereich „Evaluation“ der CRISP-DM Methology ab
        \end{itemize}\\
        \hline\hline
        6 & \textbf{Fazit} & 
        \begin{itemize}
            \item Schlussfolgerung aus der Evaluierung und Diskussion
        \end{itemize}\\
        \hline
    \end{tabularx}\\
    \end{table}