\section{Fragestellung und Zielsetzung}
Die zentrale Fragestellung, welche im Fokus dieser angestrebten 
Masterthesis steht, behandelt inwieweit das Hinzufügen von neuartigen 
immobilienwirtschaftlichen Kennwerten zu den traditionellen Faktoren 
die globale Modellgüte der Prognosen verändert und welche Faktoren 
maßgeblich dafür verantwortlich sind.

Um diese Hypothesen im Laufe der Thesis zu überprüfen, wird der 
Einfluss von traditionellen sowie den neuartigen Daten auf die 
Kaltmieten (in €/m²) von Mietobjekten der Stadt Essen analysiert. 
Dazu wird im ersten Schritt ein Datensatz bestehend aus traditionellen 
sowie neuartigen Faktoren gebildet, bereinigt und in einer explorativen 
Analyse näher betrachtet. Auf Basis des zuvor gebildeten Datensatzes 
wird mithilfe vierer Regressionsmodelle aus den Bereichen des Machine- 
sowie Deep Learning die obige Fragestellung untersucht werden und 
anhand einer Stichprobe, welche zuvor entnommen worden ist, angewendet. 
Die Erkenntnisse aus der quantitativen Untersuchung über den Einfluss, 
der angesprochenen neuartigen Faktoren, welche auf die Zielvariable 
abgeleitet wird, sowie die Prognose und die Gütekennwerte aus den 
Regressionsmodellen werden im finalen Teil dieser Studie vertieft 
sowie diskutiert.
\newpage
Im Verlauf der Arbeit sollen weitere Aspekte beleuchtet werden, um das 
Thema abzurunden:

\begin{itemize}
    \item Wie betrachtet die Kommune den aktuellen Immobilienmarkt und dessen Entwicklung?
    \item Was sind die inhaltlichen Unterschiede zwischen traditionellen und neuartigen immobilienwirtschaftlichen Kennwerten?
    \item Wie valide sind die Prognosen mit Bezug auf die Benchmarks?
    \item Kann der Grad der Sensitivität neuartiger Faktoren, die der traditionellen Faktoren übertreffen, wenn deren Dimensionen normalisiert werden?
    \item Können die neuartigen Faktoren allein die gleiche Modellgüte der traditionellen Faktoren erreichen?
\end{itemize}