%-----------------------------------
% Define document and include general packages
%-----------------------------------
% Tabellen- und Abbildungsverzeichnis stehen normalerweise nicht im
% Inhaltsverzeichnis. Gleiches gilt für das Abkürzungsverzeichnis (siehe unten).
% Manche Dozenten bemängeln das. Die Optionen 'listof=totoc,bibliography=totoc'
% geben das Tabellen- und Abbildungsverzeichnis im Inhaltsverzeichnis (toc=Table
% of Content) aus.
% Da es aber verschiedene Regelungen je nach Dozent geben kann, werden hier
% beide Varianten dargestellt.
\documentclass[12pt,oneside,titlepage,listof=totoc,bibliography=totoc]{scrartcl}
%\documentclass[12pt,oneside,titlepage]{scrartcl}

%-----------------------------------
% Dokumentensprache
%-----------------------------------
%\def\FOMEN{}% Auskommentieren um die Dokumentensprache auf englisch zu ändern
\newif\ifde
\newif\ifen

%-----------------------------------
% Meta informationen
%-----------------------------------
%-----------------------------------
% Meta Informationen zur Arbeit
%-----------------------------------

% Autor
\newcommand{\myAutor}{Dominic Hernes}

% Adresse
\newcommand{\myAdresse}{Heidestra\ss e 17 \\ \> \> \> 51147 Köln}

% Titel der Arbeit
\newcommand{\myTitel}{Analyse des Einflusses von traditionellen sowie neuartigen Faktoren auf die Prognose von Mietpreisen der Stadt Essen}

% Betreuer
\newcommand{\myBetreuer}{Prof. Dr. Frank Lehrbass}

% Lehrveranstaltung
\newcommand{\myLehrveranstaltung}{Modul Nr. 1}

% Matrikelnummer
\newcommand{\myMatrikelNr}{520014}

% Ort
\newcommand{\myOrt}{Düsseldorf}

% Datum der Abgabe
\newcommand{\myAbgabeDatum}{\today}

% Semesterzahl
\newcommand{\mySemesterZahl}{7}

% Name der Hochschule
\newcommand{\myHochschulName}{FOM Hochschule für Oekonomie \& Management}

% Standort der Hochschule
\newcommand{\myHochschulStandort}{Düsseldorf}

% Studiengang
\newcommand{\myStudiengang}{Big Data / Business Analytics}

% Art der Arbeit
\newcommand{\myThesisArt}{Exposé}

% Zu erlangender akademische Grad
\newcommand{\myAkademischerGrad}{Bachelor of Science (B.Sc.)}

% Firma
\newcommand{\myFirma}{Mustermann GmbH}


\ifdefined\FOMEN
%Englisch
\entrue
\usepackage[english]{babel}
\else
%Deutsch
\detrue
\usepackage[ngerman]{babel}
\fi


\newcommand{\langde}[1]{%
   \ifde\selectlanguage{ngerman}#1\fi}
\newcommand{\langen}[1]{%
   \ifen\selectlanguage{english}#1\fi}
\usepackage[utf8]{luainputenc}
\langde{\usepackage[babel,german=quotes]{csquotes}}
\langen{\usepackage[babel,english=british]{csquotes}}
\usepackage[T1]{fontenc}
\usepackage{fancyhdr}
\usepackage{fancybox}
\usepackage{scrlayer}
\usepackage[a4paper, left=4cm, right=2cm, top=4cm, bottom=2cm]{geometry}
\usepackage{graphicx}
\usepackage{colortbl}
\usepackage[capposition=top]{floatrow}
\usepackage{array}
\usepackage{float}      %Positionierung von Abb. und Tabellen mit [H] erzwingen
\usepackage{footnote}
% Darstellung der Beschriftung von Tabellen und Abbildungen (Leitfaden S. 44)
% singlelinecheck=false: macht die Caption linksbündig (statt zentriert)
% labelfont auf fett: (Tabelle x.y:, Abbildung: x.y)
% font auf fett: eigentliche Bezeichnung der Abbildung oder Tabelle
% Fettschrift laut Leitfaden 2018 S. 45
\usepackage[singlelinecheck=false, labelfont=bf, font=bf]{caption}
\usepackage{caption}
\usepackage{enumitem}
\usepackage{amssymb}
\usepackage{mathptmx}
%\usepackage{minted} %Kann für schöneres Syntax Highlighting genutzt werden. ACHTUNG: Python muss installiert sein.
\usepackage[scaled=0.9]{helvet} % Behebt, zusammen mit Package courier, pixelige Überschriften. Ist, zusammen mit mathptx, dem times-Package vorzuziehen. Details: https://latex-kurs.de/fragen/schriftarten/Times_New_Roman.html
\usepackage{courier}
\usepackage{amsmath}
\usepackage[table]{xcolor}
\usepackage{marvosym}			% Verwendung von Symbolen, z.B. perfektes Eurozeichen

\renewcommand\familydefault{\sfdefault}
\usepackage{ragged2e}

% Mehrere Fussnoten nacheinander mit Komma separiert
\usepackage[hang,multiple]{footmisc}
\setlength{\footnotemargin}{1em}

% todo Aufgaben als Kommentare verfassen für verschiedene Editoren
\usepackage{todonotes}

% Verhindert, dass nur eine Zeile auf der nächsten Seite steht
\setlength{\marginparwidth}{2cm}
\usepackage[all]{nowidow}

%-----------------------------------
% Farbdefinitionen
%-----------------------------------
\definecolor{darkblack}{rgb}{0,0,0}
\definecolor{dunkelgrau}{rgb}{0.8,0.8,0.8}
\definecolor{hellgrau}{rgb}{0.0,0.7,0.99}
\definecolor{mauve}{rgb}{0.58,0,0.82}
\definecolor{dkgreen}{rgb}{0,0.6,0}

%-----------------------------------
% Pakete für Tabellen
%-----------------------------------
\usepackage{epstopdf}
\usepackage{nicefrac} % Brüche
\usepackage{multirow}
\usepackage{rotating} % vertikal schreiben
\usepackage{mdwlist}
\usepackage{tabularx}% für Breitenangabe
\usepackage{array}
\usepackage{longtable}

%-----------------------------------
% sauber formatierter Quelltext
%-----------------------------------
\usepackage{listings}
% JavaScript als Sprache definieren:
\lstdefinelanguage{JavaScript}{
	keywords={break, super, case, extends, switch, catch, finally, for, const, function, try, continue, if, typeof, debugger, var, default, in, void, delete, instanceof, while, do, new, with, else, return, yield, enum, let, await},
	keywordstyle=\color{blue}\bfseries,
	ndkeywords={class, export, boolean, throw, implements, import, this, interface, package, private, protected, public, static},
	ndkeywordstyle=\color{darkgray}\bfseries,
	identifierstyle=\color{black},
	sensitive=false,
	comment=[l]{//},
	morecomment=[s]{/*}{*/},
	commentstyle=\color{purple}\ttfamily,
	stringstyle=\color{red}\ttfamily,
	morestring=[b]',
	morestring=[b]"
}

\lstset{
	%language=JavaScript,
	numbers=left,
	numberstyle=\tiny,
	numbersep=5pt,
	breaklines=true,
	showstringspaces=false,
	frame=l ,
	xleftmargin=5pt,
	xrightmargin=5pt,
	basicstyle=\ttfamily\scriptsize,
	stepnumber=1,
	keywordstyle=\color{blue},          % keyword style
  	commentstyle=\color{dkgreen},       % comment style
  	stringstyle=\color{mauve}         % string literal style
}

%-----------------------------------
%Literaturverzeichnis Einstellungen
%-----------------------------------

% Biblatex

\usepackage{url}
\usepackage{bibentry}
\urlstyle{same}

%%%% Neuer Leitfaden (2018)
\usepackage[
backend=biber,
style=ext-authoryear-ibid, % Auskommentieren und nächste Zeile kommentieren, um "Ebd." (ebenda) für sich-wiederholende Fussnoten zu nutzen
%style=ext-authoryear,
maxcitenames=3,	% mindestens 3 Namen ausgeben bevor et. al. kommt
maxbibnames=999,
mergedate=false,
date=iso,
seconds=true, %werden nicht verwendet, so werden aber Warnungen unterdrückt.
urldate=iso,
innamebeforetitle,
dashed=false,
autocite=footnote,
doi=false,
useprefix=true, % 'von' im Namen beachten (beim Anzeigen)
mincrossrefs = 1
]{biblatex}%iso dateformat für YYYY-MM-DD

%weitere Anpassungen für BibLaTex
\input{skripte/modsBiblatex2018}

%%%%% Alter Leitfaden. Ggf. Einkommentieren und Bereich hierüber auskommentieren
%\usepackage[
%backend=biber,
%style=numeric,
%citestyle=authoryear,
%url=false,
%isbn=false,
%notetype=footonly,
%hyperref=false,
%sortlocale=de]{biblatex}

%weitere Anpassungen für BibLaTex
%\input{skripte/modsBiblatex}

%%%% Ende Alter Leitfaden

%Bib-Datei einbinden
\addbibresource{literatur/literatur.bib}

% Zeilenabstand im Literaturverzeichnis ist Einzeilig
% siehe Leitfaden S. 14
\AtBeginBibliography{\singlespacing}

%-----------------------------------
% Silbentrennung
%-----------------------------------
\usepackage{hyphsubst}
\HyphSubstIfExists{ngerman-x-latest}{%
\HyphSubstLet{ngerman}{ngerman-x-latest}}{}

%-----------------------------------
% Pfad fuer Abbildungen
%-----------------------------------
\graphicspath{{./}{./abbildungen/}}

%-----------------------------------
% Weitere Ebene einfügen
%-----------------------------------
\input{skripte/weitereEbene}

%-----------------------------------
% Paket für die Nutzung von Anhängen
%-----------------------------------
\usepackage{appendix}

%-----------------------------------
% Zeilenabstand 1,5-zeilig
%-----------------------------------
\usepackage{setspace}
\onehalfspacing

%-----------------------------------
% Absätze durch eine neue Zeile
%-----------------------------------
\setlength{\parindent}{0mm}
\setlength{\parskip}{0.8em plus 0.5em minus 0.3em}

\sloppy					%Abstände variieren
\pagestyle{headings}

%----------------------------------
% Präfix in das Abbildungs- und Tabellenverzeichnis aufnehmen, statt nur der Nummerierung (siehe Issue #206).
%----------------------------------
\KOMAoption{listof}{entryprefix} % Siehe KOMA-Script Doku v3.28 S.153
\BeforeStartingTOC[lof]{\renewcommand*\autodot{:}} % Für den Doppelpunkt hinter Präfix im Abbildungsverzeichnis
\BeforeStartingTOC[lot]{\renewcommand*\autodot{:}} % Für den Doppelpunkt hinter Präfix im Tabellenverzeichnis

%-----------------------------------
% Abkürzungsverzeichnis
%-----------------------------------
\usepackage[printonlyused]{acronym}

%-----------------------------------
% Symbolverzeichnis
%-----------------------------------
% Quelle: https://www.namsu.de/Extra/pakete/Listofsymbols.pdf
\usepackage[final]{listofsymbols}

%-----------------------------------
% Glossar
%-----------------------------------
\usepackage{glossaries}
\glstoctrue %Auskommentieren, damit das Glossar nicht im Inhaltsverzeichnis angezeigt wird.
\makenoidxglossaries
\input{abkuerzungen/glossar}

%-----------------------------------
% PDF Meta Daten setzen
%-----------------------------------
\usepackage[hyperfootnotes=false]{hyperref} %hyperfootnotes=false deaktiviert die Verlinkung der Fußnote. Ansonsten inkompaibel zum Paket "footmisc"
% Behebt die falsche Darstellung der Lesezeichen in PDF-Dateien, welche eine Übersetzung besitzen
% siehe Issue 149
\makeatletter
\pdfstringdefDisableCommands{\let\selectlanguage\@gobble}
\makeatother

\hypersetup{
    pdfinfo={
        Title={\myTitel},
        Subject={\myStudiengang},
        Author={\myAutor},
        Build=1.1
    }
}

%-----------------------------------
% PlantUML
%-----------------------------------
%\usepackage{plantuml}

%-----------------------------------
% Umlaute in Code korrekt darstellen
% siehe auch: https://en.wikibooks.org/wiki/LaTeX/Source_Code_Listings
%-----------------------------------
\lstset{literate=
	{á}{{\'a}}1 {é}{{\'e}}1 {í}{{\'i}}1 {ó}{{\'o}}1 {ú}{{\'u}}1
	{Á}{{\'A}}1 {É}{{\'E}}1 {Í}{{\'I}}1 {Ó}{{\'O}}1 {Ú}{{\'U}}1
	{à}{{\`a}}1 {è}{{\`e}}1 {ì}{{\`i}}1 {ò}{{\`o}}1 {ù}{{\`u}}1
	{À}{{\`A}}1 {È}{{\'E}}1 {Ì}{{\`I}}1 {Ò}{{\`O}}1 {Ù}{{\`U}}1
	{ä}{{\"a}}1 {ë}{{\"e}}1 {ï}{{\"i}}1 {ö}{{\"o}}1 {ü}{{\"u}}1
	{Ä}{{\"A}}1 {Ë}{{\"E}}1 {Ï}{{\"I}}1 {Ö}{{\"O}}1 {Ü}{{\"U}}1
	{â}{{\^a}}1 {ê}{{\^e}}1 {î}{{\^i}}1 {ô}{{\^o}}1 {û}{{\^u}}1
	{Â}{{\^A}}1 {Ê}{{\^E}}1 {Î}{{\^I}}1 {Ô}{{\^O}}1 {Û}{{\^U}}1
	{œ}{{\oe}}1 {Œ}{{\OE}}1 {æ}{{\ae}}1 {Æ}{{\AE}}1 {ß}{{\ss}}1
	{ű}{{\H{u}}}1 {Ű}{{\H{U}}}1 {ő}{{\H{o}}}1 {Ő}{{\H{O}}}1
	{ç}{{\c c}}1 {Ç}{{\c C}}1 {ø}{{\o}}1 {å}{{\r a}}1 {Å}{{\r A}}1
	{€}{{\EUR}}1 {£}{{\pounds}}1 {„}{{\glqq{}}}1
}

%-----------------------------------
% Kopfbereich / Header definieren
%-----------------------------------
\pagestyle{fancy}
\fancyhf{}
% Seitenzahl oben, mittig, mit Strichen beidseits
% \fancyhead[C]{-\ \thepage\ -}

% Seitenzahl oben, mittig, entsprechend Leitfaden ohne Striche beidseits
\fancyhead[C]{\thepage}
%\fancyhead[L]{\leftmark}							% kein Footer vorhanden
% Waagerechte Linie unterhalb des Kopfbereiches anzeigen. Laut Leitfaden ist
% diese Linie nicht erforderlich. Ihre Breite kann daher auf 0pt gesetzt werden.
\renewcommand{\headrulewidth}{0.4pt}
%\renewcommand{\headrulewidth}{0pt}

%-----------------------------------
% Damit die hochgestellten Zahlen auch auf die Fußnote verlinkt sind (siehe Issue 169)
%-----------------------------------
\hypersetup{colorlinks=true, breaklinks=true, linkcolor=darkblack, citecolor=darkblack, menucolor=darkblack, urlcolor=darkblack, linktoc=all, bookmarksnumbered=false, pdfpagemode=UseOutlines, pdftoolbar=true}
\urlstyle{same}%gleiche Schriftart für den Link wie für den Text

%-----------------------------------
% Start the document here:
%-----------------------------------
\begin{document}

\pagenumbering{Roman}								% Seitennumerierung auf römisch umstellen
\newcolumntype{C}{>{\centering\arraybackslash}X}	% Neuer Tabellen-Spalten-Typ:
%Zentriert und umbrechbar

%-----------------------------------
% Textcommands
%-----------------------------------
\input{skripte/textcommands}

%-----------------------------------
% Titlepage
%-----------------------------------
\begin{titlepage}
	\newgeometry{left=2cm, right=2cm, top=2cm, bottom=2cm}
	\begin{center}
    \includegraphics[width=2.3cm]{abbildungen/fomLogo} \\
    \vspace{.5cm}
		\begin{Large}\textbf{\myHochschulName}\end{Large}\\
    \vspace{.5cm}
		\begin{Large}\langde{Hochschulzentrum}\langen{university location} \myHochschulStandort\end{Large}\\
		\vspace{2cm}
    \begin{Large}\textbf{\myThesisArt}\end{Large}\\
    \vspace{.5cm}
		% \langde{Berufsbegleitender Studiengang}
		% \langen{part-time degree program}\\
		% \mySemesterZahl. Semester\\
    \langde{zur Master-Thesis im Studiengang}\langen{in the study course} \myStudiengang
		\vspace{1.1cm}

		%\langde{zur Erlangung des Grades eines}\langen{to obtain the degree of}\\
    \vspace{0.5cm}
		%\begin{Large}{\myAkademischerGrad}\end{Large}\\
		% Oder für Hausarbeiten:
		%\textbf{im Rahmen der Lehrveranstaltung}\\
		%\textbf{\myLehrveranstaltung}\\
		\vspace{1.8cm}
		\langde{über das Thema}
		\langen{on the subject}\\
    \vspace{0.5cm}
		\large{\textbf{\myTitel}}\\
		\vspace{2cm}
    \langde{von}\langen{by}\\
    \vspace{0.5cm}
    \begin{Large}{\myAutor}\end{Large}\\
	\end{center}
	\normalsize
	\vfill
    \begin{tabular}{ l l }
        \langde{Betreuer} % für Hausarbeiten
        %\langde{Erstgutachter} % für Bachelor- / Master-Thesis
        \langen{Advisor}: & \myBetreuer\\
        \langde{Matrikelnummer}
        \langen{Matriculation Number}: & \myMatrikelNr\\
        \langde{Abgabedatum}
        \langen{Submission}: & \myAbgabeDatum
    \\
    \end{tabular}
\end{titlepage}


%-----------------------------------
% Inhaltsverzeichnis
%-----------------------------------
% Um das Tabellen- und Abbbildungsverzeichnis zu de/aktivieren ganz oben in Documentclass schauen
%\setcounter{page}{2}
%\addtocontents{toc}{\protect\enlargethispage{-20mm}}% Die Zeile sorgt dafür, dass das Inhaltsverzeichnisseite auf die zweite Seite gestreckt wird und somit schick aussieht. Das sollte eigentlich automatisch funktionieren. Wer rausfindet wie, kann das gern ändern.
%\setcounter{tocdepth}{4}
%\tableofcontents
%\newpage

%-----------------------------------
% Abbildungsverzeichnis
%-----------------------------------
%\listoffigures
%\newpage
%-----------------------------------
% Tabellenverzeichnis
%-----------------------------------
%\listoftables
%\newpage
%-----------------------------------
% Abkürzungsverzeichnis
%-----------------------------------
% Falls das Abkürzungsverzeichnis nicht im Inhaltsverzeichnis angezeigt werden soll
% dann folgende Zeile auskommentieren.
%\addcontentsline{toc}{section}{\abbreHeadingName}
%\input{abkuerzungen/acronyms}
%\newpage

%-----------------------------------
% Symbolverzeichnis
%-----------------------------------
% In Overleaf führt der Einsatz des Symbolverzeichnisses zu einem Fehler, der aber ignoriert werdne kann
% Falls das Symbolverzeichnis nicht im Inhaltsverzeichnis angezeigt werden soll
% dann folgende Zeile auskommentieren.
%\addcontentsline{toc}{section}{\symheadingname}
\input{skripte/symbolDef}
%listofsymbols
%\newpage

%-----------------------------------
% Glossar
%-----------------------------------
%\printnoidxglossaries
%\newpage

%-----------------------------------
% Sperrvermerk
%-----------------------------------
%\input{kapitel/anhang/sperrvermerk}

%-----------------------------------
% Seitennummerierung auf arabisch und ab 1 beginnend umstellen
%-----------------------------------
\pagenumbering{arabic}
\setcounter{page}{1}

%-----------------------------------
% Kapitel / Inhalte
%-----------------------------------
% Die Kapitel werden über folgende Datei eingebunden
% Hinzugefügt aufgrund von Issue 167
%-----------------------------------
% Kapitel / Inhalte
%-----------------------------------
\section{Problemstellung}
In der Vergangenheit wurde der Wert einer Immobilie durch eine 
Immobilienfirma sowie die Bereitschaft eine bestimmte Miete für 
eine Immobilie durch eine Privatperson zu zahlen an den direkten 
Gegebenheiten der Immobilie selbst und deren unmittelbarer Umgebung 
gemessen. Die Lage einer Immobilie war das höchste Gut bei dem 
Bemessen einer Miete. Jedoch hat sich dieses Verhalten in den 
letzten Jahren verändert, durch das hohe Angebot von etwaigen 
Bewertungsportalen und Kartenwerkzeugen im Internet, wie 
beispielsweise Google, Yelp oder Tripadvisor, welche eine neue 
Sicht auf die makroskopische Betrachtung einer Immobilie 
ermöglichen. Mit diesen Werkzeugen lässt sich heutzutage die 
Lebensqualität messen, welche im Zusammenspiel mit der Lage dem 
potenziellen Mieter etliche neue Variablen für seine Betrachtung 
mitgeben. So kann dieser den Grad der Bereitschaft zum Zahlen einer 
gewissen Miete konkreter bilden und durch den erweiterten Blick auf 
die frei zugängliche makroskopische Ebene neu entwickeln. Zu dem 
kann sich der Wert der Immobilie je nach Zielgruppe sowie 
Quartiersentwicklung unter diesen Variablen sehr differenziert 
weiterentwickeln. Diesen Einfluss der neuartigen Variablen, 
welche auch als nicht-traditionellen Kennwerte bezeichnet werden, 
wollen wir in dieser Arbeit untersuchen.

\section{Fragestellung und Zielsetzung}
Die zentrale Fragestellung, welche im Fokus dieser angestrebten 
Masterthesis steht, behandelt inwieweit das Hinzufügen von neuartigen 
immobilienwirtschaftlichen Kennwerten zu den traditionellen Faktoren 
die globale Modellgüte der Prognosen verändert und welche Faktoren 
maßgeblich dafür verantwortlich sind.

Um diese Hypothesen im Laufe der Thesis zu überprüfen, wird der 
Einfluss von traditionellen sowie den neuartigen Daten auf die 
Kaltmieten (in €/m²) von Mietobjekten der Stadt Essen analysiert. 
Dazu wird im ersten Schritt ein Datensatz bestehend aus traditionellen 
sowie neuartigen Faktoren gebildet, bereinigt und in einer explorativen 
Analyse näher betrachtet. Auf Basis des zuvor gebildeten Datensatzes 
wird mithilfe vierer Regressionsmodelle aus den Bereichen des Machine- 
sowie Deep Learning die obige Fragestellung untersucht werden und 
anhand einer Stichprobe, welche zuvor entnommen worden ist, angewendet. 
Die Erkenntnisse aus der quantitativen Untersuchung über den Einfluss, 
der angesprochenen neuartigen Faktoren, welche auf die Zielvariable 
abgeleitet wird, sowie die Prognose und die Gütekennwerte aus den 
Regressionsmodellen werden im finalen Teil dieser Studie vertieft 
sowie diskutiert.
\newpage
Im Verlauf der Arbeit sollen weitere Aspekte beleuchtet werden, um das 
Thema abzurunden:

\begin{itemize}
    \item Wie betrachtet die Kommune den aktuellen Immobilienmarkt und dessen Entwicklung?
    \item Was sind die inhaltlichen Unterschiede zwischen traditionellen und neuartigen immobilienwirtschaftlichen Kennwerten?
    \item Wie valide sind die Prognosen mit Bezug auf die Benchmarks?
    \item Kann der Grad der Sensitivität neuartiger Faktoren, die der traditionellen Faktoren übertreffen, wenn deren Dimensionen normalisiert werden?
    \item Können die neuartigen Faktoren allein die gleiche Modellgüte der traditionellen Faktoren erreichen?
\end{itemize}
%\newpage
\section{Methodik}
Der Aufbau dieser Thesis erfolgt nach Vorbild des Vertiefungsmodells 
nach Phillip Mayring, welcher Bestandteile der quantitativen und 
qualitativen Analysen zu einem kombinierten Gesamtmodell vereint\footcite[Vgl. ][]{Mayring2001}.  
Auf Basis des Vertiefungsmodells wird die quantitative Untersuchung 
in Anlehnung an dem CRISP-DM durchgeführt, einem Standartprozessmodell 
zur Modellierung von Data Mining Projekten im Bereich der 
computerunterstützen Analyse\footcite[Vgl. ][]{Wirth2000}. Die aus dieser Untersuchung 
ermittelten Ergebnisse werden daraufhin vertieft, verglichen sowie diskutiert und 
bilden die qualitative Untersuchung ab.

Im ersten Teil dieser Thesis wird der aktuelle Stand der kommunalen Stadtentwicklung 
sowie der Entwicklung des Wohnungsbedarfs der Stadt Essen dargestellt\footcite[Vgl. ][]{StadtEssen2018}. 
Zudem werden die traditionellen und neuartigen Faktoren definiert und deren Unterschiede 
beschrieben, welche für die Beantwortung der zentralen Fragestellung mitunter 
untersucht werden\footcite[Vgl. ][]{Asaftei2018}.

Die Beschreibung, Aufbereitung und Zusammenführung der Datensätze, bestehend aus den 
verschiedenen Faktoren der traditionellen und neuartigen immobilienwirtschaftlichen 
Faktoren erfolgt im zweiten Teil. Der Hauptschwerpunkt dieses Arbeitsschrittes, liegt 
neben der Beschreibung aller genutzten Datensätze, in der explorativen Datenanalyse sowie 
deren Aufarbeitung für den Einsatz in den Prognosemodellen. Die Aufbereitung umfasst 
verschiedene Methoden von der Datenbereinigung\footcite[Vgl. ][]{Rahm2000}, bis hin zur 
Transformationsmethoden\footcite[Vgl. ][]{Box1964} sowie Encoding Varianten\footcite[Vgl. ][]{Cerda2018}, 
um diese bestmöglich für die Analyse vorzubereiten. Im letzten Abschnitt des zweiten 
Teils wird der gesamte Datensatz für das Training in den Modellen 
vorbereitet. Dieser wird für diesen Zweck in drei Teile aufgeteilt. Einem Trainings- 
sowie Testdatensatz, welche für den Einsatz an den Modellen genutzt werden wird und 
einem zufällig auserwählten Validierungsdatensatz.

Im letzten Teil der Analyse werden nunmehr Prognosemodelle auf Basis der zuvor 
aufgearbeiteten Daten eingesetzt und durchgeführt, indem diese mithilfe des Training- sowie Testdatensatzes 
trainiert werden, so dass diese an dem Validierungsdatensatz angewendet werden können.
Die Optimierung der Hyperparameter, der genutzten Modelle erfolgt mit Hilfe der Random 
Search Methode\footcite[Vgl. ][]{Bergstra2012}. Die Prognosen des Validierungsdatensatzes 
bilden dadurch einen essenziellen Teil der Diskussion ab. 

Folgende Prognosemodelle kommen für die Analyse zum Einsatz:

\begin{itemize}
    \item Lineare Regression\footcite[Vgl. ][]{Verbeek2017}
    \item Random Forest\footcite[Vgl. ][]{Breiman2001}
    \item XGBoost\footcite[Vgl. ][]{Chen2016}
    \item Multilayer Perceptron mit mehr als zwei Ebenen\footcite[Vgl. ][]{Brause1991, Goodfellow2017} 
\end{itemize}

Die Effizienz dieser Modelle werden anhand ihres Gütemaßes dem adjusted R² (kurz adj. R²) 
sowie der Verlustmessung Root Mean Squared Error (kurz RSME) bewertet\footcite[Vgl. ][]{Goodfellow2017,Larose2015}.
Zudem können auf Basis der erstellten Prognosen die einflussreichsten Faktoren aller 
Modelle entnommen werden. Die Effizienz sowie die Faktoren der Modelle, welche einen 
hohen Einfluss die Prognose haben sind weitere Bestandteile der Ergebnisdiskussion neben 
den Prognosen.
\newpage
Validiert werden die Ergebnisse anhand zweier Benchmarks. Diese sind zu einem der 
ortsübliche Mietspiegel\footcite[Vgl. ][]{AmtfurStadterneuerungundBodenmanagement-StadtEssen2020}, 
der durch die Stadt Essen erhoben worden ist, sowie den 
ermittelten Angebotsmieten der Zeiträume zweites Quartal 2018 bis erstes Quartal 2019, 
welcher durch die CBRE, einen internationalen Dienstleister aus der Immobilienwirtschaft, 
im LEG-Wohnungsmarktreport 2019 erhoben worden sind \footcite[Vgl. ][]{LEG2019}.  

Nach der Analyse werden die Prognosen, die Effizienz der Modelle sowie die einflussreichsten 
Faktoren im fünften Teil diese Thesis in der Ergebnisdiskussion vertieft, indem diese 
übersichtlich aufgearbeitet und dementsprechend gegenübergestellt werden. 

\section{Kommentiertes Literaturverzeichnis}
\fullcite{AmtfurStadterneuerungundBodenmanagement-StadtEssen2020}
\begin{itemize}
    \item Der Mietspiegel 2020 der Stadt Essen ist ein Leitfaden zur 
    Ermittlung von Mietrichtwerten der Stadt Essen und deren 
    Wohnimmobilien.
\end{itemize}

\fullcite{Asaftei2018}
\begin{itemize}
    \item Diese Untersuchung, welche durch McKinsey \& Company ausgeführt
    worden ist, behandelt die nicht-traditionellen immobilienwirtschaftliche
    Kennwerte und betrachtet diese auf dem amerikanischen Immobilienmarkt. 
\end{itemize}

\fullcite{Bergstra2012}
\begin{itemize}
    \item Der Random Search ist ein Vorgehen, welches die optimalsten
    Parameter eines Modells ermittelt mit dem Ziel die besten Schätzer
    zu erhalten. Durch diesen Algorithmus werden vorher festgelegte
    Reichenweiten von Parametern durchgetestet und anhand einer
    Verlustmessung bewertet und auserwählt.
\end{itemize}
\newpage
\fullcite{Box1964}
\begin{itemize}
    \item Die BoxCox-Transformation ist eine beliebte Methode für die
    Transformation von Daten in eine Normalverteilung. In dieser
    wissenschaftlichen Arbeit wird die Umsetzung dieser Methodik 
    beschrieben und mit anderen Tranformationmethoden verglichen.
\end{itemize}

\fullcite{Brause1991}
\begin{itemize}
    \item Rüdiger Brause beschreibt in diesem Buch die 
    naturwissenschaftlichen Grundlagen von neuronalen Netzen und 
    die daraus entstandenen Modellarchitekturen in der Informatik.
\end{itemize}

\fullcite{Breiman2001}
\begin{itemize}
    \item Der Random Forest, welcher ein gängiger Algorithmus aus 
    dem Bereich des Machine Learning ist wird in dieser 
    wissenschaftlicher Arbeit in Gänze beschrieben. Dieser Algorithmus 
    schätzt die Prognose mit Hilfe von Entscheidungsbäumen.
\end{itemize}

\fullcite{Cerda2018}
\begin{itemize}
    \item Wenn in einem Datensatz kategoriale Daten vorhanden sind, ist es für 
    diverse Analysemethoden notwendig, dass diese in angepasste Merkmalsvektoren 
    (eng. feature vector) in der Regel in nominale Werte umgeschrieben werden.
    Dies wird über Encodingmethoden ermöglicht, welche in dieser Untersuchung
    beschrieben werden.
\end{itemize}

\fullcite{Chen2016}
\begin{itemize}
    \item Der XGBoost Algorithmus ist eine moderne Erweiterung der 
    Entscheidungsbäume/Ensemble für Prognosen im Bereich des Machine Learning und 
    erfreut sich hoher Beliebtheit. Tianqi Chen und Carlos Guestrin sind die Erfinder 
    des Algorithmus und beschreiben diesen in dieser Arbeit in voller Gänze.
\end{itemize}
\newpage
\fullcite{Goodfellow2017}
\begin{itemize}
    \item Dieses Buch behandelt die Thematik rund um Deep Learning. In diesem werden die
    mathematischen und konzeptionellen Hintergrunde, relevante Konzepte der 
    linearen Algebra und Wahrscheinlichkeitstheorie beschrieben, welche gebraucht werden um neuronalen
    Netze aufzubauen. Neben diesen Grundlagen geht dieses Buch noch weiter und beschreibt zu dem
    die Optimierung und Regulierung der diversen Arten von neuronale Netzen.
\end{itemize}

\fullcite{Larose2015}
\begin{itemize}
    \item Data Mining findet täglich mehr Verwendung und Akzeptanz, denn es ermöglicht Unternehmen 
    profitable Muster und Trends aus ihren bestehenden Datenbanken aufzudecken und diese zu
    nutzen. Daniel T. und Chantal D. Larose beschreiben verschiedene Analysemethoden für
    ein effizientes Data Mining.
\end{itemize}

\fullcite{LEG2019}
\begin{itemize}
    \item Die LEG Immobilien AG bringt in regelmäßigen Abständen in Zusammenarbeit mit der 
    CBRE den LEG-Wohnungsmarktreport heraus. Dieser Report analysiert die aktuelle Situation 
    der Wohnimmobilienmärkte bezogen auf die Großstädte der Bundesrepublik Deutschland.
\end{itemize}

\fullcite{Mayring2001}
\begin{itemize}
    \item In dieser Arbeit stellt Phillip Mayring verschiedene Hybridmodelle für empirische
    Untersuchungen vor. Das Vertiefungsmodell, welcher der Rahmen dieser Thesis ist wird
    darin beschrieben.
\end{itemize}

\fullcite{Rahm2000}
\begin{itemize}
    \item Die Datenbereinigung gehört zu den zeitintensivsten 
    Schritten in der Modellentwicklung. In dieser Arbeit werden Methoden 
    und Ansätze zur Datenbereinigung für wiederkehrende Probleme beschrieben. 
\end{itemize}
\newpage
\fullcite{StadtEssen2018}
\begin{itemize}
    \item Dieses Gutachten, welches durch die Stadt Essen in Auftrag gegeben worden ist, 
    behandelt die Frage nach dem Wohnungsangebot sowie -nachfrage der Stadt Essen in den kommenden Jahren 
    und skizziert mögliche Handlungsfelder um den steigenden Bedarf zu minimieren. 
\end{itemize}

\fullcite{Verbeek2017}
\begin{itemize}
    \item Dieses Buch dient als Leitfaden für alternative Techniken in der Ökonometrie mit 
    Schwerpunkt auf Intuition und der praktischen Umsetzung dieser Ansätze. Es deckt 
    ein breiten Themenspektrum ab einschliesslich der linearen Regression mit dessen 
    Diagnostik, Zeitreihenanalysen und Paneldatenanalysen. 
\end{itemize}

\fullcite{Wirth2000}
\begin{itemize}
    \item Diese wissenschaftliche Arbeit beschreibt den kompletten CRISP-DM Prozess 
    mit allen zugehörigen Prozessschritten und deren Inhalten. 
\end{itemize}
\newpage
\section{Vorläufige Gliederung}

\begin{table}[H]
    \begin{tabularx}{\textwidth}[ht]{|l|X|X|}
      \hline
      \textbf{Nr.} & \textbf{Kapitel} & \textbf{Behandelt}\\
      \hline
      \hline
        1 & \textbf{Einleitung}

        \begin{itemize}
            \item Motivation
            \item Problemstellung
            \item Zielsetzung
        \end{itemize} 
        
        & \begin{itemize}
            \item Ziel- und Problembeschreibung des Themas
        \end{itemize}\\
        \hline\hline
        2 & \textbf{Der Immobilienmarkt in Essen, NRW} 
        
        \begin{itemize}
            \item 2.1 Aktuelle und prognostizierte kommunale Entwicklungen der Mietsituation 
            und des Mietbedarfes der Stadt Essen
            \item 2.2 Beschreibung der Unterschiede der traditionellen und 
            nicht-traditionellen Faktoren, welche die Grundlage der Analyse bilden
        \end{itemize}

        & \begin{itemize}
            \item Grundlagenbeschreibung
            \item Spiegelt den Bereich „Business Understanding“ der CRISP-DM Methology ab
        \end{itemize}\\
        \hline\hline
        3 & \textbf{Der Datensatz und dessen Aufbereitung} 
        
        \begin{itemize}
            \item 3.1 Beschreibung der Datensätze sowie der Benchmarks, welche zur 
            Analyse genutzt werden
            \item 3.2 Beschreibung der Datenbereinigung und Zusammenschluss der 
            traditionellen und nicht-traditionellen Faktoren
            \item 3.3 Explorative Datenanalyse
            \item 3.4 Präparation der Daten für die Anwendungen zur Prognose
        \end{itemize}
        
        & \begin{itemize}
            \item Beschreibung und Aufarbeitung aller genutzten Daten, welche die 
            Basis für die Prognosen bilden
            \item -	Spiegelt die Bereiche „Data Understanding“ sowie 
            „Data Preparation“ der CRISP-DM Methology ab
        \end{itemize}\\
        \hline
        \multicolumn{3}{r}{\textit{Fortführung auf der nächsten Seite}} \\
    \end{tabularx}\\
    \end{table}
\newpage
\begin{table}[H]
    \begin{tabularx}{\textwidth}[ht]{|l|X|X|}
      \hline
      \textbf{Nr.} & \textbf{Kapitel} & \textbf{Behandelt}\\
      \hline
      \hline
        4 & \textbf{Modellierung und Analyse} 
        
        \begin{itemize}
            \item 4.1 Beschreibung der verwendeten Modelle sowie deren Einsatz zur 
            Prognose der Mieten 
            \begin{itemize}
                \item Lineare Regression
                \item Random Forest
                \item XGBoost
                \item Multilayer Perceptron
            \end{itemize}
            \item 4.2 Vergleich der Güte sowie der Verlustmessungen
        \end{itemize}
        
        & \begin{itemize}
            \item Weiterverarbeitung aller genutzten Daten, sowie eine erste 
            Analyse, durch den Einsatz der Prognosemodelle
            \item Spiegelt den Bereich „Modeling“ der CRISP-DM Methology ab
        \end{itemize}\\
        \hline\hline
        5 & \textbf{Evaluierung und Diskussion} 
        
        \begin{itemize}
            \item Aufbereitung der Validierungsergebnisse mit den verschiedenen 
            Prognosen sowie Gegenüberstellung mit dem Benchmark
            \item Betrachtung der effizientesten Faktoren, welche durch die 
            Prognosemodelle für die Validierung genutzt worden sind
            \item Diskussion über die Ergebnisse
        \end{itemize}

        & \begin{itemize}
            \item Behandelt die Validierung und diskutiert diese mit Betrachtung auf 
            die genutzten Prognosemodelle und vergleicht diese mit den Benchmarks, sowie 
            deren Faktoren
            \item Spiegelt den Bereich „Evaluation“ der CRISP-DM Methology ab
        \end{itemize}\\
        \hline\hline
        6 & \textbf{Fazit} & 
        \begin{itemize}
            \item Schlussfolgerung aus der Evaluierung und Diskussion
        \end{itemize}\\
        \hline
    \end{tabularx}\\
    \end{table}
%\input{kapitel/kapitel_1/kapitel_1}
%\input{kapitel/kapitel_2/kapitel_2}
%\input{kapitel/fazit/fazit}


%-----------------------------------
% Apendix / Anhang
%-----------------------------------
%\newpage
%\section*{\AppendixName} %Überschrift "Anhang", ohne Nummerierung
%\addcontentsline{toc}{section}{\AppendixName} %Den Anhang ohne Nummer zum Inhaltsverzeichnis hinzufügen

%\begin{appendices}
% Nachfolgende Änderungen erfolgten aufgrund von Issue 163
%\makeatletter
%\renewcommand\@seccntformat[1]{\csname the#1\endcsname:\quad}
%\makeatother
%\addtocontents{toc}{\protect\setcounter{tocdepth}{0}} %
%	\renewcommand{\thesection}{\AppendixName\ \arabic{section}}
%	\renewcommand\thesubsection{\AppendixName\ \arabic{section}.\arabic{subsection}}
%	\input{kapitel/anhang/anhang}
%\end{appendices}
%\addtocontents{toc}{\protect\setcounter{tocdepth}{2}}

%-----------------------------------
% Literaturverzeichnis
%-----------------------------------
\newpage

% Die folgende Zeile trägt ALLE Werke aus literatur.bib in das
% Literaturverzeichnis ein, egal ob sie zietiert wurden oder nicht.
% Der Befehl ist also nur zum Test der Skripte sinnvoll und muss bei echten
% Arbeiten entfernt werden.
%\nocite{*}

%\addcontentsline{toc}{section}{Literatur}

% Die folgenden beiden Befehle würden ab dem Literaturverzeichnis wieder eine
% römische Seitennummerierung nutzen.
% Das ist nach dem Leitfaden nicht zu tun. Dort steht nur dass 'sämtliche
% Verzeichnisse VOR dem Textteil' römisch zu nummerieren sind. (vgl. S. 3)
%\pagenumbering{Roman} %Zähler wieder römisch ausgeben
%\setcounter{page}{4}  %Zähler manuell hochsetzen

% Ausgabe des Literaturverzeichnisses

% Keine Trennung der Werke im Literaturverzeichnis nach ihrer Art
% (Online/nicht-Online)
%\begin{RaggedRight}
%\printbibliography
%\end{RaggedRight}

% Alternative Darstellung, die laut Leitfaden genutzt werden sollte.
% Dazu die Zeilen auskommentieren und folgenden code verwenden:

% Literaturverzeichnis getrennt nach Nicht-Online-Werken und Online-Werken
% (Internetquellen).
% Die Option nottype=online nimmt alles, was kein Online-Werk ist.
% Die Option heading=bibintoc sorgt dafür, dass das Literaturverzeichnis im
% Inhaltsverzeichnis steht.
% Es ist übrigens auch möglich mehrere type- bzw. nottype-Optionen anzugeben, um
% noch weitere Arten von Zusammenfassungen eines Literaturverzeichnisse zu
% erzeugen.
% Beispiel: [type=book,type=article]
\printbibliography[nottype=online,heading=bibintoc,title={\langde{Literaturverzeichnis}\langen{Bibliography}}]

% neue Seite für Internetquellen-Verzeichnis
\newpage

% Laut Leitfaden 2018, S. 14, Fussnote 44 stehen die Internetquellen NICHT im
% Inhaltsverzeichnis, sondern gehören zum Literaturverzeichnis.
% Die Option heading=bibintoc würde die Internetquelle als eigenen Eintrag im
% Inhaltsverzeicnis anzeigen.
%\printbibliography[type=online,heading=bibintoc,title={\headingNameInternetSources}]
%\printbibliography[type=online,heading=subbibliography,title={\headingNameInternetSources}]

%\input{kapitel/anhang/erklaerung}
\end{document}
